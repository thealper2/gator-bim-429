\section{Literatür Özeti}

\indent Deloitte, Türkiye'deki dizi sektörü hakkında kapsamlı bir analiz sunmaktadır. Altı büyük yayıncı kuruluşun (ATV, FOX TV, Kanal D, Show TV, Star TV ve TRT 1) yayın akışları incelenmiş ve 2010-2014 yılları arasındaki dört yayın sezonu boyunca yayınlanan dizilerin reytingleri, süreleri ve maliyetleri hakkında bilgi verilmiştir. \\

\indent Türkiye'deki dizi sektörünün hızla büyüdüğü ve giderek daha fazla dizi üretildiği belirtilmektedir. 2010-2014 yılları arasında, önde gelen altı kanalda yayınlanan dizi sayısı ortalama olarak 70-80 arasında değişmiştir. Bu dönemde, en az bir dizisi önde gelen TV kanallarında yayınlanan 85 civarında yapım şirketi bulunmaktadır. Ancak, bu yapım şirketlerinin yaklaşık \%50'si bu dönem içerisinde sadece bir dizi yayınladıkları için oldukça parçalı bir pazar yapısı göze çarpmaktadır. \\

\indent Türkiye'deki dizi sektörünün büyümesinde, 2000'li yılların başından beri yaşanan endüstriyel, ekonomik ve sanatsal değişimlerin etkili olduğu belirtilmektedir. Bu değişimler, Türkiye'deki dizi sektörünün uluslararası alanda daha fazla tanınmasına ve ihracatının artmasına yol açmıştır. Türk dizilerinin Ortadoğu, Balkanlar, Kuzey Afrika ve Latin Amerika gibi birçok ülkede popüler olduğu ve Türk dizilerinin yurt dışında büyük bir izleyici kitlesi tarafından takip edildiği belirtilmektedir. \\

\indent Türkiye'deki dizi sektörünün büyümesine rağmen, sektör paydaşlarının nitelikli veriye olan ihtiyacının kapsamlı bir doküman ile karşılanması gerektiği belirtilmektedir. Sektör paydaşlarının ihtiyaç duyduğu nitelikli verileri sunarak, Türkiye'deki dizi sektörü hakkında kapsamlı bir analiz sunmaktadır. \\

\indent Türkiye'deki dizi sektörünün maliyetleri hakkında da bilgi verilmektedir. Ortalama olarak bir dizi, reklamlar ve tekrarlar dahil yaklaşık 150-180 dakika sürmektedir. Batı standartları ile karşılaştırıldığında bu süre oldukça yüksektir. Yapım şirketleri, ünlü aktörlerin, senaristlerin ve yapımcıların tercih edilmesi nedeniyle maliyetlerini artırmaktadır. Buna karşılık, RTÜK düzenlemeleri bir dizi yayını içerisinde yer alacak reklam kuşaklarının sayısını ve sürelerini kısıtlamaktadır; bu da yayıncı kuruluşları dizi sürelerini uzatmaya teşvik etmektedir. \\

\indent Dizilerin tekrarlarına ve özetlerine haftalık program akışlarında önemli bir pay verildiği ve bu tekrarların yayıncı kuruluşlar için önemli ek gelir kaynağı yarattığı belirtilmektedir. Ancak, bazı durumlardaki yayın adedi sınırlamalarına rağmen genellikle TV kanalları tekrarlar için yapımcılara herhangi bir ücret ödememektedirler. \\

\indent Dizilerin izleyiciler tarafından en fazla tercih edilen program tipi olduğu ve ilk beş programın yaklaşık \%50-55'inin dizilerden oluştuğu belirtilmektedir. Ancak, reyting performansı düşük olan dizilerin genellikle sonlandırıldığı ve yayınlanan dizilerin yayınlandıkları günlerde ortalama olarak ilk 10'a giremedikleri belirtilmektedir. \\

\indent Türkiye'deki dizi sektörünün rekabetçi yapısı incelenmektedir. İrili ufaklı pek çok yapımcı kuruluşun sayıca az yayıncı kuruluşun kısıtlı yayın saatleri için rekabet etmelerinin yanı sıra sektördeki firmaların kurumsallaşma açısından nispeten daha emekleme döneminde olmaları, yapımcı kuruluşların sürdürülebilir bir yapıda faaliyet göstermelerinin önündeki en büyük engellerden birisidir. Reyting verileri incelendiğinde televizyon pazarının hakimiyeti paylaşılan bir yapıya sahip olduğu gözlemlenmektedir; yapım şirketleri az sayıdaki TV kanalına ürünlerini pazarlamaktadırlar. TV kanalları, reytingler ile reklam gelirleri arasındaki doğrudan ilişki nedeniyle, yüksek reytingli dizileri tercih etmektedirler. \\

\indent Türkiye'deki dizi sektörünün yurt dışı pazarlarda rekabet gücünün artırılması için öneriler sunulmaktadır. Bu öneriler arasında, Türk dizilerinin yurt dışında daha fazla tanıtılması, yurt dışı pazarlarda daha fazla Türk dizisi yayınlanması, Türk dizilerinin yurt dışında daha fazla ödül kazanması ve Türk dizilerinin yurt dışında daha fazla festivale katılması yer almaktadır. \\

\indent Sonuç olarak, Türkiye'deki dizi sektörünün hızla büyüdüğünü ve giderek daha fazla dizi üretildiğini göstermektedir. Ancak, artan maliyetler ve rekabet nedeniyle, düşük reytingli dizilerin genellikle sonlandırıldığı ve yayınlanan dizilerin yayınlandıkları günlerde ortalama olarak ilk 10'a giremedikleri belirtilmektedir. Türkiye'deki dizi sektörünün rekabetçi yapısı, maliyetleri, tekrarları ve özetleri, yurt dışı pazarlarda rekabet gücü ve diğer konular hakkında kapsamlı bir analiz sunulmaktadır.